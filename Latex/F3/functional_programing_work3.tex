\documentclass[12pt, a4paper]{article}
\usepackage{graphicx}
\usepackage{amsmath}
\usepackage{amssymb}
\graphicspath{ {./images/} }
\title{
  \begin{figure}[th]
    \centering
    \includegraphics[width=0.2\textwidth]{Univalle}
  \end{figure}
  \textbf{Universidad del Valle
    \\{\Large Facultad de ingeniería}
  \\{\large Ingeniería en sistemas}}}
\author{Cristian David Pacheco Torres
  \\ 2227437
  \\ Juan Sebastián Molina Cuéllar
  \\ 2224491}
\date{\today}
\begin{document}
\maketitle
 { Taller 3: Reconocimiento de patrones.}
\newpage{}
\tableofcontents
\newpage{}
\section{Maniobras en trenes.}
\subsection{Aplicar movimiento.}
\subsubsection{Informe de uso del reconocimiento de patrones.}
\textbf{Tabla 1}
\subsubsection{Informe de Corrección.}
\subsubsection{Conclusión.}
\subsection{Aplicar movimientos}
\subsubsection{Informe de uso del reconocimiento de patrones.}
\textbf{Tabla 2}
\subsubsection{Informe de Corrección.}
\subsubsection{Conclusión.}
\subsection{Definir maniobras}
\subsubsection{Informe de uso del reconocimiento de patrones.}
\textbf{Tabla 3}
\subsubsection{Informe de Corrección.}
Sea k $\in N$, $ 0 \leq k \leq n$, un entero que indica el número actual de maniobras en el trayecto principal $T_p$,
$S_0$ = $<b_1, \ldots,~b_i,~\ldots, b_{n-1}, b_n>$ una secuencia que define el estado inicial en $T_p$;
$S_1 = < c_1, c_2,~\ldots~, b_{n-1}, c_n >$, $ 0 \leq j \leq n - 1$ elementos, una secuencia que define el estado de un trayectto secundario $T_2$ en el paso $k$,
$S_2 =  < e_1, e_2,~\ldots~, e_{n-1}, e_n >$, $ 0 \leq j \leq n - 1$ elementos, una secuencia que define el estado de un trayectto secundario $T_2$ en el paso $k$;
una función l que determina el número de elementos de la secuencia $s$ de entrada;
y $P_{k-n}^{k-n}(S_k)$ una funcion de permutacion de $k-n$ en $k-n$ elementos sobre los elementos de secuencia
$S$ en el paso $k$ de la maniobra. \\
Por premisa se tiene que $S_0[i] = Sd[j]$,  para $0 \leq i \leq n$ y $0 \leq j \leq n$ \\
Se quiere demostar que $\exists~ S_n = < a_1, a_2, ~\ldots~, a_{n-1}, a_n >~|~S_n[i] = Sd[i]$ para $0 \leq i \leq n$

Se define  \\

\begin{itemize}
  \item Un estado $s=(S_k, S_1, S_2, S_d, m)$ donde
  \begin{itemize}
    \item $S_k$ representa la secuencia en la iteracion $k$.
    \item $S_1$ el estado sobre el trayecto $T_1$.
    \item $S_2$ el estado sobre el trayecto $T_2$.
    \item $Sd$ el estado deseado.
    \item $m$ la lista de moviemientos (maniobras) hasta el paso $k$.
  \end{itemize}
  \item El estado inicial $s_0=(S_0, [~~], [~~], S_d, [~~])$.
  \item $S_f=(S_n, [~~], [~~],~S_d,~ m)$. donde $S_n[i] = S_d[i]$ en $0 \leq i \leq n$.
  \item $Inv(S_k, S_1, S_2, S_d, m)\equiv$ $ \exists~p := P_{n-1}^{n-1} ( Si~k = 0 \rightarrow S_0 \lor S_1  ++ S_2) \\ |~S_k[k:n] = S_d[k:n]$ $~\land~l(m) \leq (n-1) + \ldotp~+ (n-k+1)$ \\
  para $0 \leq k \leq n$.
  \item $transformar(S_k, S_1, S_2, S_d, m)$ = \[   
    T(S_k, S_1, S_2, S_d, m) = 
    \begin{cases}
       \text{$(S_k + c_0, S_1 - [c_0], S_2,  S_d, m + (Uno(-1)))$,} \\ ~~~~~~~~~~~~~~~~~~~~~~~~~~~~~~~~~~~ \text{Si $ k \neq 0 \land k \neq n \land~$$S_1[0] = S_d[k]$}\\
       \text{$(S_k + e_0, S_1, S_2 - [e_0],  S_d, m + (Dos(-1)))$,} \\ ~~~~~~~~~~~~~~~~~~~~~~~~~~~~~~~~~~~ \text{Si $ k \neq 0 \land k \neq n \land~$$S_2[0] = S_d[k]$}\\
       \text{$([~~], , [b_1, \ldots,~b_i,~\ldots, b_{n-1}, b_n], [~~],  S_d, [~~] + (Uno(-n)))$,}  \\ ~~~~~~~~~~~~~~~~~~~~~~~~~~~~~~~~~~~ \text{Si $ k = 0$}\\
       \text{$([~~], , [~~], [~~],  S_d, [~~] + (Uno(-n)))$,}  \\ ~~~~~~~~~~~~~~~~~~~~~~~~~~~~~~~~~~~ \text{Si $ k = n$}\\
     \end{cases}
\]
\end{itemize}

\subsubsection{Conclusión.}

\end{document}