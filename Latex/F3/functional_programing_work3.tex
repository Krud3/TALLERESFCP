\documentclass[12pt, a4paper]{article}
\usepackage{graphicx}
\usepackage{amsmath}
\usepackage{amssymb}
\graphicspath{ {./images/} }
\title{
  \begin{figure}[th]
    \centering
    \includegraphics[width=0.2\textwidth]{Univalle}
  \end{figure}
  \textbf{Universidad del Valle
    \\{\Large Facultad de ingeniería}
  \\{\large Ingeniería en sistemas}}}
\author{Cristian David Pacheco Torres
  \\ 2227437
  \\ Juan Sebastián Molina Cuéllar
  \\ 2224491}
\date{\today}
\begin{document}
\maketitle
 { Taller 3: Reconocimiento de patrones.}
\newpage{}
\tableofcontents
\newpage{}
\section{Maniobras en trenes.}
\subsection{Aplicar movimiento.}
\subsubsection{Informe de uso del reconocimiento de patrones.}
\textbf{Tabla 1}
\subsubsection{Informe de Corrección.}
\subsubsection{Conclusión.}
\subsection{Aplicar movimientos}
\subsubsection{Informe de uso del reconocimiento de patrones.}
\textbf{Tabla 2}
\subsubsection{Informe de Corrección.}
\subsubsection{Conclusión.}
\subsection{Definir maniobras}
\subsubsection{Informe de uso del reconocimiento de patrones.}
\textbf{Tabla 3}
\subsubsection{Informe de Corrección.}
Sea k $\in N$, $ 0 \leq k \leq n$, un entero que indica el número actual de maniobras en el trayecto principal $T_p$,
$S_0$ = $<b_1, \ldots,~b_i,~\ldots, b_{n-1}, b_n>$ una secuencia que define el estado inicial en $T_p$;
$S_1 = < c_1, c_2,~\ldots~, b_{n-1}, c_n >$, $ 0 \leq j \leq n - 1$ elementos, una secuencia que define el estado de un trayectto secundario $T_2$ en el paso $k$,
$S_2 =  < e_1, e_2,~\ldots~, e_{n-1}, e_n >$, $ 0 \leq j \leq n - 1$ elementos, una secuencia que define el estado de un trayectto secundario $T_2$ en el paso $k$;
una función l que determina el número de elementos de la secuencia $s$ de entrada;
y $P_{k-n}^{k-n}(S_k)$ una funcion de permutacion de $k-n$ en $k-n$ elementos sobre los elementos de secuencia
$S$ en el paso $k$ de la maniobra. \\
Por premisa se tiene que $S_0[i] = Sd[j]$,  para $0 \leq i \leq n$ y $0 \leq j \leq n$ \\
Se quiere demostar que $\exists~ S_n = < a_1, a_2, ~\ldots~, a_{n-1}, a_n >~|~S_n[i] = Sd[i]$ para $0 \leq i \leq n$

Por tanto, se tiene \\

\begin{itemize}
  \item Un estado $s=(S_k, S_1, S_2, S_d, m)$ donde
  \begin{itemize}
    \item $S_k$ representa la secuencia en la iteracion $k$.
    \item $S_1$ el estado sobre el trayecto $T_1$.
    \item $S_2$ el estado sobre el trayecto $T_2$.
    \item $Sd$ el estado deseado.
    \item $m$ la lista de moviemientos (maniobras) hasta el paso $k$.
  \end{itemize}
  \item El estado inicial $S_0=(S_0, [~~], [~~], S_d, [~~])$.
  \item $S_f=(S_n, [~~], [~~],~S_d,~ m)$. donde $S_n[i] = S_d[i]$ en $0 \leq i \leq n$.
  \item $Inv(S_k, S_1, S_2, S_d, m)\equiv$ $ \exists~p := P_{n-k}^{n-k} ( Si~k = 0 \rightarrow S_0 \lor S_1  + S_2) \\ |~p[n-k:n] = S_d[n-k:n]$ $~\land~l(m) \leq (n-1) +~\ldots~+ (n-k+1), \\ \land ~S_d[i] = T(S_k)[i],~~\forall i~|~0 \leq i \leq n-k \land 0 \leq k \leq n$ \\
  
  \item $transformar(S_k, S_1, S_2, S_d, m)$ = \[   
    T(S_k, S_1, S_2, S_d, m) = 
    \begin{cases}
       \text{$((S_k[k:n],~S_1,~S_2),~S_d[k:n],~m )$,} \\ ~~~~~~~~~~~~~~~~~~~~~~ \text{Si $ k \neq 0 \land k \neq n \land~$$S_k[0] = S_d[0]$}\\
       \text{$(S_k,~[b_k] + S_1,~S_2), ~S_d,~m + (Uno(1))$,}  \\ ~~~~~~~~~~~~~~~~~~~~~~ \text{Si $S_k \neq [~] \land S_K = (S_k,~S_1,~[~~]) \land l(S_k) = 1$}\\
       \text{$(S_k, [~~], [~~]),  S_d,~[~~] + (Uno(-n))$,}  \\ ~~~~~~~~~~~~~~~~~~~~~~~ \text{Si $ k \neq 0 \land k \neq n \land~$$S_k[0] \neq S_d[0]$}\\
       \text{$(S_k,~S_1,~[b_k, \ldots, b_n]), ~S_d,~m + (Dos(k-n))$,}  \\ ~~~~~~~~~~~~~~~~~~~~~~ \text{Si $S_k \neq [~] \land S_K = (S_k,~S_1,~[~~]) \land l(S_1) = 1$}\\
       \text{$(S_k,~S_1[l(S_1)-1:l(S_1)],~S_2), ~S_d, ~m + (Uno(-l(S_1)-1))$,}  \\ ~~~~~~~~~~~~~~~~~~~~~~ \text{Si $S_k \neq [~] \land S_K = (S_k,~S_1,~[~~])$}\\
       \text{$(S_k + [c_1],~[~],~S_2), ~S_d, ~m + (Uno(-1))$,}  \\ ~~~~~~~~~~~~~~~~~~~~~~ \text{Si $S_k \neq [~] \land S_K = (S_k,~S_1,~S_2) \land l(S_1) = 1$}\\
       \text{$(S_k + [e_1],~S_1,~[~]), ~S_d, ~m + (Dos(-1))$,}  \\ ~~~~~~~~~~~~~~~~~~~~~~ \text{Si $S_k \neq [~] \land S_K = (S_k,~S_1,~S_2) \land l(S_2) = 1$}\\
       \text{$([~] + [b_k:b_{n-1}] + S_1,~S_2), ~S_d, ~m + (Uno(l(S_k)))$,}  \\ ~~~~~~~~~~~~~~~~~~~~~~ \text{Si $S_k \neq [~] \land S_K = (S_k,~[~],~S_2) \land l(S_2) = 1$}\\
       \text{$(S_k + [e_k:e_{n-1}],~[~],~[~e_1]), ~S_d, ~m + (Dos(-l(S_2)-1))$,}  \\ ~~~~~~~~~~~~~~~~~~~~~~ \text{Si $S_k \neq [~] \land S_K = (S_k,~[~],~S_2) $}\\
       \text{$((S_K, S_1, S_2), ~S_d, ~m )$,}~~~~~~~ \text{$ Si~S_k = (\_,\_,\_)$}\\
     \end{cases} \]
\end{itemize}

Ahora se procede a demostrar la correctitud de los enunciados anteriores. Para el estado inicial $S_0$, se tiene: \\

$Inv(S_0) \rightarrow Inv(S_0,~[~],~[~], S_d,~[~]),$ para $k = 0$ iteraciones se cumple que existe
una permutacion $p = P_n^n(S_0)$, tal que lleva $S_0$ a $S_d$, pues $S_0~\subseteq S_d \land S_d \subseteq S_0$. \\
Por otro lado, $S_k = S_0$, el estado no se ha modificado mediante la transformacion $T$. Ádemas, $l(m) = 0 \leq k-n$ se cumple. \\
$( S_k \neq S_f \land Inv(S_k)) \rightarrow Inv(T(S_{k+1}))$ \\
$\equiv$ \\
$(\exists~p := P_{n-k}^{n-k} ( S_1  + S_2)~|~p[n-k:n] = S_d[n-k:n]$ $~\land~l(m) \leq (n-1) +~\ldots~+ (n-k+1), \land ~S_d[i] = T(S_k)[i],para~ i~|~0 \leq i \leq n-k)$ \\
$(\exists~p := P_{n-k+1}^{n-k+1} ([S_1+ S_2]-[a \in S_1 + S_2])~|~p[n-k+1:n] = S_d[n-k+1:n]$ $~\land~l(m) \leq (n-1) +~\ldots~+ (n-k+2), \land ~S_d[i] = T(S_{k+1})[i],para~ i~|~0 \leq i \leq n-k+1)$ \\
Y, por ultimo, se tiene para $S_f$ \\
$Inv(S_f) \rightarrow S_n = S_d$ \\
$\equiv$ \\
$(\exists~p := P_0^0 ([~])~|~p[~] = S_d[~]$ $~\land~l(m) \leq \frac{n(n-1)}{2} \land ~S_d[i] = S_{i}, ~para~0 \leq i \leq n)$ \\
$\equiv$ \\
$true~\land~true$ \\
$\equiv$ \\
$true$
\subsubsection{Conclusión.}

\end{document}