\documentclass[12pt, a4paper]{article}
\usepackage{graphicx}
\usepackage{amsmath}
\usepackage{amssymb}
\usepackage{float}
\usepackage{listings}
\usepackage{xcolor}  

\lstset{
    language=Scala,  % lenguaje de programación
    basicstyle=\ttfamily\small,  % estilo de la fuente del código
    numbers=left,  % posición de los números de línea
    numberstyle=\tiny\color{gray},  % estilo de los números de línea
    stepnumber=1,  % el paso entre dos números de línea
    numbersep=5pt,  % cómo de lejos están los números de línea del código
    backgroundcolor=\color{white},  % color de fondo
    showspaces=false,  % mostrar espacios añadiendo particular subrayado
    showstringspaces=false,  % subrayar espacios dentro de las cadenas
    showtabs=false,  % mostrar tabs dentro de las cadenas añadiendo particular subrayado
    frame=single,  % añade un marco al código
    rulecolor=\color{black},  % si no se establece, el color del marco será el del texto
    tabsize=2,  % tamaño de los tabs
    captionpos=b,  % posición del título de la tabla (t=top, b=bottom)
    breaklines=true,  % establecer si las líneas automáticas se rompen
    breakatwhitespace=false, 
    escapeinside={\%*}{*)},  % si quieres añadir LaTeX dentro de tu código
    keywordstyle=\color{blue},  % estilo de las palabras clave
    commentstyle=\color{olive},  % estilo de los comentarios
    stringstyle=\color{violet},  % estilo de las cadenas
}

\graphicspath{ {./images/} }
\title{
  \begin{figure}[th]
    \centering
    \includegraphics[width=0.2\textwidth]{Univalle}
  \end{figure}
  \textbf{Universidad del Valle
    \\{\Large Facultad de ingeniería}
  \\{\large Ingeniería en sistemas}}}
\author{Cristian David Pacheco Torres
  \\ 2227437
  \\ Juan Sebastián Molina Cuéllar
  \\ 2224491}
\date{\today}
\begin{document}
\maketitle
 { Taller 3: Reconocimiento de patrones.}
\newpage{}
\tableofcontents
\newpage{}
\section{Maniobras en trenes.}
\subsection{Aplicar movimiento.}
\subsubsection{Informe de uso del reconocimiento de patrones.}
\begin{table}[H]
  \scriptsize
 \begin{tabular}{ |p{4cm}|p{3cm}|p{5.5cm}|  }
  \hline
  \multicolumn{3}{|c|}{Reconocimiento de patrones} \\
  \hline
  \textbf{Función}& ¿Se utilizó pattern Matching?  & ¿Razón?\\
  \hline
   $AplicarMovimiento$ & Sí &  La función utiliza $pattern~matching$ para analizar el tipo de Movimiento y determinar la lógica de transición de estados. Dependiendo del movimiento y su magnitud, se ajustan las posiciones de los elementos en las tres sub-listas del estado, garantizando que el estado resultante refleje adecuadamente el movimiento aplicado. \\
   \hline
 \end{tabular}
 \centering
 \caption{Tabla para la función $AplicarMovimiento$}
 \end{table}
 \begin{lstlisting}[caption=Código en Scala para la funcion aplicarMovimiento, label=lst:scala_code]
  def aplicarMovimiento(state:Estado, mov:Movimiento): Estado = {
      mov match {
          case Uno(n) if(n > 0) => (state._1.dropRight(n),  state._1.takeRight(n) ::: state._2, state._3)
          case Dos(n) if(n > 0) => (state._1.dropRight(n), state._2, state._1.takeRight(n) ::: state._3)
          case Uno(n) if(n < 0) => (state._1 ::: state._2.take(-n),  state._2.drop(-n), state._3)
          case Dos(n) if(n < 0) => (state._1 ::: state._3.take(-n), state._2, state._3.drop(-n))
          case _  => (state._1, state._2, state._3)
      }
  }
  \end{lstlisting}
  La función \textbf{aplicarMovimiento} toma dos argumentos: un estado $(state)$ y un movimiento $(mov)$, y retorna un nuevo estado. Utiliza el reconocimiento de patrones para analizar el tipo y las propiedades del $mov$ proporcionado y, dependiendo de este, aplica diferentes lógicas para modificar el state y generar un nuevo estado. Los estados y los movimientos están representados como tuplas y casos, respectivamente, y la función se asegura de manejar diferentes escenarios posibles (movimientos positivos/negativos y de dos tipos diferentes Uno y Dos) para actualizar el estado de manera adecuada. En caso de que el movimiento no coincida con ninguno de los patrones definidos, el estado se mantiene sin cambios.
  
\subsubsection{Informe de Corrección.}
\textbf{Argumentación sobre la corrección:} \\
$\forall$ $s$ $\in$ $Estado$ : $Estado$ $\{Tren1,Tren2,Tren3\}$ $\wedge$  $Tren$ $\{List[Vagon]\}$ $\wedge$ $Vagon\{Any\}$\\
$\forall$ $m$ $\in$ $Movimiento$ : $Movimiento$ $\{Uno(Int),Dos(Int)\}$ $\wedge$ $Int$ $\{\mathbb{Z}\}$\\
\\ \textbf{Caso base:}
\begin{itemize}
  \item $if~(m = (Uno(Int) \vee Dos(Int)) \wedge Int = 0 )\rightarrow case \_ ~~Estado(Tren1,Tren2,Tren3)$
\end{itemize}
 \textbf{Caso posibles:}
\begin{itemize}
  \item $if~(m = (Uno(Int)) \wedge Int > 0 )\rightarrow \\ Estado(Tren1.dropRight(Int),Tren1.takeRight(Int):::Tren2,Tren3)$
  \item $ \text{Si}~(m = (\text{Dos}(n)) \wedge n > 0 ) \rightarrow \\ \text{Estado}(Tren1.\text{dropRight}(n), Tren2, Tren1.\text{takeRight}(n):::Tren3)$
  \item $\text{Si}~(m = (\text{Uno}(n)) \wedge n < 0 ) \rightarrow \\ \text{Estado}(Tren1:::Tren2.\text{take}(-n), Tren2.\text{drop}(-n), Tren3)$ 
  \item $\text{Si}~(m = (\text{Dos}(n)) \wedge n < 0 ) \rightarrow \\ \text{Estado}(Tren1:::Tren3.\text{take}(-n), Tren2, Tren3.\text{drop}(-n))$ 
\end{itemize}
\textbf{Caso prueba:} \\ \\
Si tenemos: \\ $val state = (List(a, b, c), List(d), List(e, f)) \wedge val mov = Uno(2)$
\\ \\ Remplazando : \\ $aplicarMovimiento(state, mov)= \\ aplicarMovimiento((List(a, b, c), List(d), List(e, f)), Uno(2))$ \\ \\
Por el caso "$Uno(Int) \wedge Int > 0$" : \\ = $(List(a, b, c).dropRight(2),  List(a, b, c).takeRight(2) ::: List(d), List(e, f))$
\\ $= (List(a),  List( b, c, d), List(e, f))$
\\ \\
\newpage{}
\textbf{Casos de prueba: } \\
\begin{lstlisting}[caption=Casos de prueba para la funcion aplicarMovimiento, label=lst:scala_code]
  val e6 = (List('a', 'b', 'c','d'), List(), List()) 
  val e7 = aplicarMovimiento(e6, Uno(2)) //Esperado: (List('a', 'b'), List('c','d'), List())
  val e8 = aplicarMovimiento(e7, Dos(1)) //Esperado: (List('a'), List('c','d'), List('b'))
  val e9 = aplicarMovimiento(e8, Uno(-1)) //Esperado: (List('a','c'), List('d'), List('b'))
  val e10 = aplicarMovimiento(e9, Dos(-1)) //Esperado: (List('a','c', 'b'), List('d'), List())
  val e11 = aplicarMovimiento(e10, Uno(1)) //Esperado: (List('a','c'), List('b','d'), List())
  val e12 = aplicarMovimiento(e11, Dos(2)) //Esperado: (List(), List('b','d'), List('a','c'))
  \end{lstlisting}
\subsubsection{Conclusión.}
La utilización de un modelo de sustitución y casos de prueba para validar la corrección del algoritmo $aplicarMovimiento$ fue esencial para asegurar su funcionalidad en diversos escenarios y movimientos. El modelo de sustitución permite descomponer y entender la lógica detrás de cada movimiento y transición de estado, mientras que los casos de prueba aseguran que el algoritmo actúa de manera esperada en situaciones específicas y límites. Juntos, proporcionan un marco robusto para verificar la integridad, la robustez y la fiabilidad del algoritmo, garantizando que los movimientos y estados resultantes sean coherentes y precisos.
\subsection{Aplicar movimientos}
\subsubsection{Informe de uso del reconocimiento de patrones.}
\begin{table}[H]
  \scriptsize
 \begin{tabular}{ |p{4cm}|p{3cm}|p{5.5cm}|  }
  \hline
  \multicolumn{3}{|c|}{Reconocimiento de patrones} \\
  \hline
  \textbf{Función}& ¿Se utilizó pattern Matching?  & ¿Razón?\\
  \hline
  $AplicarMovimientos$  & Sí &  La función emplea $pattern~matching$ para descomponer recursivamente la lista de movimientos ($maniobras$) en cabeza y cola. A través de esta descomposición, aplica secuencialmente cada movimiento al estado actual, acumulando y retornando una lista de todos los estados intermedios generados durante la aplicación de la secuencia de movimientos.\\
  \hline
 \end{tabular}
 \centering
 \caption{Tabla para la función $AplicarMovimientos$}
 \end{table}
 \begin{lstlisting}[caption=Código en Scala para la funcion aplicarMovimientos, label=lst:scala_code]
  def aplicarMovimientos(state:Estado, movs:Maniobra): List[Estado] = {
    def helper(state:Estado, movs:Maniobra, acc:List[Estado]): List[Estado] = movs match {
        case Nil => acc
        case x::xs => {
            val newState = aplicarMovimiento(state, x)  
            helper(newState, xs, acc :+ newState)
        }
    }
    helper(state, movs, List())   
  }
  \end{lstlisting}
  La función $\textbf{aplicarMovimientos}$ toma un $estado$ y una lista de $movimientos$, y devuelve una lista de $estados$ resultantes de aplicar esos $movimientos$ secuencialmente. Utiliza una función auxiliar $helper$ para realizar esta tarea de manera recursiva, aplicando cada movimiento al $estado$ actual y acumulando los $estados$ intermedios en una lista, que se devuelve cuando todos los $movimientos$ han sido aplicados.
  \newpage{}
  \subsubsection{Informe de Corrección.}

  $s \forall \in Estado : Estado \{Tren1,Tren2,Tren3\} \wedge  Tren \{List[Vagon]\} \wedge Vagon\{Any\}$ \\
  $m \forall \in Maniobra : Maniobra \{List[Movimiento]\} \wedge Movimiento \{Uno(Int),Dos(Int)\} \wedge Int \{\mathbb{Z}\}$ \\
  \textbf{Caso inicial:} \\
  \begin{itemize}
    \item $S_0 = (s_{s_0}, List(mov_0,..,mov_n),List() )$
    \[ mov \in Movimiento \wedge i,e,c,n \in \mathbb{Z^+}\]
    \item $S = (S_{s_k} , List(mov_k,..,mov_n),List(List(Estado_0), ...,List(Estado_k)) )$
  \item $S_f = (s_{s_n}, List(),List(Estado_0), ...,List(Estado_n)) $
  \item $Inv(S_{s_k} , m_k, e_k) \equiv e_k \subset \{s_1 \cup s_2 \cup s_3\} ~	\cup ~ {aplicarMovimiento(m_k)} $
    \item $Transformar~ ((s_1,s_2,s_3), m_k, e_k ) = ( \{s_1 \cup s_2 \cup s_3\} ~	\cup ~ {aplicarMovimiento(m_k)}, \\ m_{k1}, e_{k} + aplicarMovimiento(e_{k}))$ donde $(s_1,s_2,s_3)$ define el estado de la estacion, $m_k$, y $e_k$ es la lista de movimientos restantes y estados aplicados hasta la iteracion $K$, respectivamente.
\end{itemize} 
\textbf{Hipótesis Inductiva (HI):}

Asumimos que para un estado arbitrario \(s_k\) y una lista de movimientos \(m_k\) de longitud \(n-k\), la función \texttt{aplicarMovimientos} produce una lista de estados \(e_k\) tal que cada estado en \(e_k\) es el resultado de aplicar los movimientos en \(m_k\) al estado inicial \(s_0\).

\[ HI(k) \equiv aplicarMovimientos(s_k, m_k) = \{e_k\} \cup \{s_{k}\}\]

\textbf{Paso Inductivo:}

Queremos mostrar que si la hipótesis inductiva es verdadera para \(k\), entonces también lo es para \(k+1\).

\[ HI(k+1) \equiv aplicarMovimientos(s_{k+1}, m_{k+1}) = \{e_{k+1}\} \cup \{s_{k+1}\}\]

Considerando un movimiento  \(m_{k}\) y el estado  \(s_{k+1}\) que es el resultado de aplicar \(mov_{k}\) a \(s_k\), queremos mostrar que:

\[ aplicarMovimientos(s_{k}, m_k ) = \{e_{k+1} \}\cup \{s_{k+1}\} \]

Por la definición de la función \texttt{aplicarMovimientos}, sabemos que:

\[ aplicarMovimientos(s_k \cup \{e_{k}\}, m_{k+1} ) = aplicarMovimientos(s_{k+1}, m_{k+1}) \cup \{e_{k+1}\} \]

Por la hipótesis inductiva, sabemos que:

\[ aplicarMovimientos(s_{k}, m_k) = e_k \]

Por lo tanto, podemos concluir que:

\[ aplicarMovimientos(s_k, m_k) = e_{k+1} \cup \{s_{k+1}\} \]

Esto completa el paso inductivo y, por lo tanto, la corrección de la función \texttt{aplicarMovimientos} ha sido demostrada por inducción. Y asá cada vez la lista se va volviendo más pequeña y va actualizando el estado con los movimientos que va consumiendo hasta que la lista de movimientos se vacia. \\


  
\textbf{Casos de prueba: } \\
\begin{lstlisting}[caption=Para ver los valores esperados por favor referirse al archivo de pruebas.sc. Debido a que los resultados son tan extensos que dañan la estructura del documento., label=lst:scala_code]
  val e13 = (List('a', 'b', 'c'), List('d'), List('e', 'f'))
  aplicarMovimientos(e13, List(Uno(2), Dos(1)))
  aplicarMovimientos(e13, List(Uno(-1), Dos(2), Uno(1)))
  aplicarMovimientos(e13, List(Dos(1), Uno(-1), Dos(-1)))
  val e14 = (List('x', 'y'), List('z', 'a', 'b'), List('c'))
  aplicarMovimientos(e14, List(Dos(2), Uno(1)))
  aplicarMovimientos(e14, List(Uno(-1), Dos(-2), Uno(2)))
  \end{lstlisting}
\subsubsection{Conclusión.}
Su correctitud la garantiza la inducción estructural, debido a que  se utiliza la recursión para llamarse a si misma. La inducción estructural nos muestra que de una lista de movmimientos al aplicarlos consecutivamente, nos llevará a nuevos estdos que representaran los movimientos de un tren y sus diferentes trayectorias, hasta acabar con la posibilidad de moverse.

En cuanto al uso de pattern matching en algoritmos, este enfoque permite una estructuración clara y legible del código, facilitando la identificación y manejo de diversos casos y subcasos de manera ordenada y explícita.En particular en este problema, determinar el movimiento siguiente a realizar y comprobar cuándo ha llegado al final de la lista de movimientos dada.
\subsection{Definir maniobras}
\subsubsection{Informe de uso del reconocimiento de patrones.}
\begin{table}[H]
  \scriptsize
 \begin{tabular}{ |p{4cm}|p{3cm}|p{5.5cm}|  }
  \hline
  \multicolumn{3}{|c|}{Reconocimiento de patrones} \\
  \hline
  \textbf{Función}& ¿Se utilizó pattern Matching?  & ¿Razón?\\
  \hline 
  $ DefinirManiobra$  & Sí &  La función utiliza $pattern~matching$ para descomponer el estado deseado y el estado actual en sus componentes respectivos, permitiendo la comparación elemento por elemento entre ellos. A través de una serie de reglas y condiciones, se generan movimientos que transforman el estado inicial en el estado deseado, considerando las diferencias entre los elementos y sus posiciones en las respectivas $sub-listas$.\\
  \hline
 \end{tabular}
 \centering
 \caption{Tabla para la función $DefinirManiobra$}
 \end{table}
 \begin{lstlisting}[caption=Código en Scala para la funcion aplicarMovimientos, label=lst:scala_code]
  def definirManiobra(initialState:Tren,wantedState:Tren):Maniobra = {
    def moveHelper(state: Estado, wantedState: Tren, movs: List[Movimiento]): List[Movimiento] = {
        wantedState match {
            case Nil => movs
            case x::xs if(!state._1.isEmpty && x == state._1.head) => {
                moveHelper((state._1.tail,state._2,state._3), xs, movs)
            }
            case x::xs => {
                ...ver codigo en package.scala....                    
            }
            case _ => movs
        }
    }
    moveHelper((initialState, List(), List()), wantedState, List())
}
  \end{lstlisting}
  La función \textbf{definirManiobra} busca determinar una secuencia de movimientos $(maniobra)$ que transforme un estado inicial de un tren $(initialState)$ en un estado deseado $(wantedState)$. Utiliza una función auxiliar $moveHelper$ que, mediante el uso de pattern matching, evalúa y compara los elementos de los estados actual y deseado, generando los movimientos necesarios para alinear cada componente del tren en la posición deseada, y acumulando estos movimientos en una lista que se devuelve como resultado.
\subsubsection{Informe de Corrección.}
Sea k $\in N$, $ 0 \leq k \leq n$, un entero que indica el número actual de maniobras en el trayecto principal $T_p$,
$S_0$ = $<b_1, \ldots,~b_i,~\ldots, b_{n-1}, b_n>$ una secuencia que define el estado inicial en $T_p$;
$S_1 = < c_1, c_2,~\ldots~, b_{n-1}, c_n >$, $ 0 \leq j \leq n - 1$ elementos, una secuencia que define el estado de un trayectto secundario $T_2$ en el paso $k$,
$S_2 =  < e_1, e_2,~\ldots~, e_{n-1}, e_n >$, $ 0 \leq j \leq n - 1$ elementos, una secuencia que define el estado de un trayectto secundario $T_2$ en el paso $k$;
una función l que determina el número de elementos de la secuencia $s$ de entrada;
y $P_{k-n}^{k-n}(S_k)$ una funcion de permutacion de $k-n$ en $k-n$ elementos sobre los elementos de secuencia
$S$ en el paso $k$ de la maniobra. \\
Por premisa se tiene que $S_0[i] = Sd[j]$,  para $0 \leq i \leq n$ y $0 \leq j \leq n$ \\
Se quiere demostar que $\exists~ S_n = < a_1, a_2, ~\ldots~, a_{n-1}, a_n >~|~S_n[i] = Sd[i]$ para $0 \leq i \leq n$\\

<<<<<<< HEAD
Por tanto, se tiene \\
=======
 Por tanto, se tiene \\
>>>>>>> 7a9250ddb606ef27e04cf2956c3de8f85afa3a00

\begin{itemize}
  \item Un estado $s=(S_k, S_1, S_2, S_d, m)$ donde
  \begin{itemize}
    \item $S_k$ representa la secuencia en la iteracion $k$.
    \item $S_1$ el estado sobre el trayecto $T_1$.
    \item $S_2$ el estado sobre el trayecto $T_2$.
    \item $Sd$ el estado deseado.
    \item $m$ la lista de moviemientos (maniobras) hasta el paso $k$.
  \end{itemize}
  \item El estado inicial $S_0=(S_0, [~~], [~~], S_d, [~~])$.
  \item $S_f=(S_n, [~~], [~~],~S_d,~ m)$. donde $S_n[i] = S_d[i]$ en $0 \leq i \leq n$.
  \item $Inv(S_k, S_1, S_2, S_d, m)\equiv$ $ \exists~p := P_{n-k}^{n-k} ( Si~k = 0 \rightarrow S_0 \lor S_1  + S_2) \\ |~p[n-k:n] = S_d[n-k:n]$ $~\land~l(m) \leq (n-1) +~\ldots~+ (n-k+1), \\ \land ~S_d[i] = T(S_k)[i],~~\forall i~|~0 \leq i \leq n-k \land 0 \leq k \leq n$ \\
  
  \item $transformar(S_k, S_1, S_2, S_d, m)$ = \[   
    T(S_k, S_1, S_2, S_d, m) = 
    \begin{cases}
       \text{$((S_k[k:n],~S_1,~S_2),~S_d[k:n],~m )$,} \\ ~~~~~~~~~~~~~~~~~~~~~~ \text{Si $ k \neq 0 \land k \neq n \land~$$S_k[0] = S_d[0]$}\\
       \text{$(S_k,~[b_k] + S_1,~S_2), ~S_d,~m + (Uno(1))$,}  \\ ~~~~~~~~~~~~~~~~~~~~~~ \text{Si $S_k \neq [~] \land S_K = (S_k,~S_1,~[~~]) \land l(S_k) = 1$}\\
       \text{$(S_k, [~~], [~~]),  S_d,~[~~] + (Uno(-n))$,}  \\ ~~~~~~~~~~~~~~~~~~~~~~~ \text{Si $ k \neq 0 \land k \neq n \land~$$S_k[0] \neq S_d[0]$}\\
       \text{$(S_k,~S_1,~[b_k, \ldots, b_n]), ~S_d,~m + (Dos(k-n))$,}  \\ ~~~~~~~~~~~~~~~~~~~~~~ \text{Si $S_k \neq [~] \land S_K = (S_k,~S_1,~[~~]) \land l(S_1) = 1$}\\
       \text{$(S_k,~S_1[l(S_1)-1:l(S_1)],~S_2), ~S_d, ~m + (Uno(-l(S_1)-1))$,}  \\ ~~~~~~~~~~~~~~~~~~~~~~ \text{Si $S_k \neq [~] \land S_K = (S_k,~S_1,~[~~])$}\\
       \text{$(S_k + [c_1],~[~],~S_2), ~S_d, ~m + (Uno(-1))$,}  \\ ~~~~~~~~~~~~~~~~~~~~~~ \text{Si $S_k \neq [~] \land S_K = (S_k,~S_1,~S_2) \land l(S_1) = 1$}\\
       \text{$(S_k + [e_1],~S_1,~[~]), ~S_d, ~m + (Dos(-1))$,}  \\ ~~~~~~~~~~~~~~~~~~~~~~ \text{Si $S_k \neq [~] \land S_K = (S_k,~S_1,~S_2) \land l(S_2) = 1$}\\
       \text{$([~] + [b_k:b_{n-1}] + S_1,~S_2), ~S_d, ~m + (Uno(l(S_k)))$,}  \\ ~~~~~~~~~~~~~~~~~~~~~~ \text{Si $S_k \neq [~] \land S_K = (S_k,~[~],~S_2) \land l(S_2) = 1$}\\
       \text{$(S_k + [e_k:e_{n-1}],~[~],~[~e_1]), ~S_d, ~m + (Dos(-l(S_2)-1))$,}  \\ ~~~~~~~~~~~~~~~~~~~~~~ \text{Si $S_k \neq [~] \land S_K = (S_k,~[~],~S_2) $}\\
       \text{$((S_K, S_1, S_2), ~S_d, ~m )$,}~~~~~~~ \text{$ Si~S_k = (\_,\_,\_)$}\\
     \end{cases} \]
\end{itemize}

Ahora se procede a demostrar la correctitud de los enunciados anteriores. Para el estado inicial $S_0$, se tiene: \\
<<<<<<< HEAD

$Inv(S_0) \rightarrow Inv(S_0,~[~],~[~], S_d,~[~]),$ para $k = 0$ iteraciones se cumple que existe
una permutacion $p = P_n^n(S_0)$, tal que lleva $S_0$ a $S_d$, pues $S_0~\subseteq S_d \land S_d \subseteq S_0$. \\
Por otro lado, $S_k = S_0$, el estado no se ha modificado mediante la transformacion $T$. Ádemas, $l(m) = 0 \leq k-n$ se cumple. \\
$( S_k \neq S_f \land Inv(S_k)) \rightarrow Inv(T(S_{k+1}))$ \\
$\equiv$ \\
$(\exists~p := P_{n-k}^{n-k} ( S_1  + S_2)~|~p[n-k:n] = S_d[n-k:n]$ $~\land~l(m) \leq (n-1) +~\ldots~+ (n-k+1), \land ~S_d[i] = T(S_k)[i],para~ i~|~0 \leq i \leq n-k)$ \\
$(\exists~p := P_{n-k+1}^{n-k+1} ([S_1+ S_2]-[a \in S_1 + S_2])~|~p[n-k+1:n] = S_d[n-k+1:n]$ $~\land~l(m) \leq (n-1) +~\ldots~+ (n-k+2), \land ~S_d[i] = T(S_{k+1})[i],para~ i~|~0 \leq i \leq n-k+1)$ \\
Y, por ultimo, se tiene para $S_f$ \\
$Inv(S_f) \rightarrow S_n = S_d$ \\
$\equiv$ \\
$(\exists~p := P_0^0 ([~])~|~p[~] = S_d[~]$ $~\land~l(m) \leq \frac{n(n-1)}{2} \land ~S_d[i] = S_{i}, ~para~0 \leq i \leq n)$ \\
$\equiv$ \\
$true~\land~true$ \\
$\equiv$ \\
$true$
\subsubsection{Conclusión.}
=======
>>>>>>> 7a9250ddb606ef27e04cf2956c3de8f85afa3a00

$Inv(S_0) \rightarrow Inv(S_0,~[~],~[~], S_d,~[~]),$ para $k = 0$ iteraciones se cumple que existe
una permutacion $p = P_n^n(S_0)$, tal que lleva $S_0$ a $S_d$, pues $S_0~\subseteq S_d \land S_d \subseteq S_0$. \\
Por otro lado, $S_k = S_0$, el estado no se ha modificado mediante la transformacion $T$. Ádemas, $l(m) = 0 \leq k-n$ se cumple. \\
\[( S_k \neq S_f \land Inv(S_k)) \rightarrow Inv(T(S_{k+1})) \equiv \]\\

$(\exists~p := P_{n-k}^{n-k} ( S_1  + S_2)~|~p[n-k:n] = S_d[n-k:n]$ $~\land~l(m) \leq (n-1) +~\ldots~+ (n-k+1), \land ~S_d[i] = T(S_k)[i],para~ i~|~0 \leq i \leq n-k) \rightarrow$ \\
$(\exists~p := P_{n-k+1}^{n-k+1} ([S_1+ S_2]-[a \in S_1 + S_2])~|~p[n-k+1:n] = S_d[n-k+1:n]$ $~\land~l(m) \leq (n-1) +~\ldots~+ (n-k+2), \land ~S_d[i] = T(S_{k+1})[i],para~ i~|~0 \leq i \leq n-k+1)$ \\ \\
Y, por ultimo, se tiene para $S_f$ \\
\[Inv(S_f) \rightarrow S_n = S_d \equiv\] \\
$(\exists~p := P_0^0 ([~])~|~p[~] = [~]$ $~\land~l(m) \leq \frac{n(n-1)}{2} \land ~S_d[i] = S_{n}[i], ~para~0 \leq i \leq n)$  \\
\[
\equiv
\]
\[true~\land~true~\land~true  \equiv true \]


\textbf{Casos de prueba: } \\
\begin{lstlisting}[caption=Para ver los valores esperados por favor referirse al archivo de pruebas.sc. Debido a que los resultados son tan extensos que dañan la estructura del documento., label=lst:scala_code]
  definirManiobra(List('a', 'b', 'c'), List('c', 'b', 'a'))
  definirManiobra(List('w', 'x', 'y', 'z', 'a'), List('a', 'z', 'y', 'x', 'w'))
  definirManiobra(List('m', 'n'), List('n', 'm'))
  definirManiobra(List('g', 'h', 'i', 'j', 'k', 'l'), List('l', 'k', 'j', 'i', 'h', 'g'))
  definirManiobra(List('t', 'u', 'v'), List('v', 'u', 't'))
  \end{lstlisting}
\subsubsection{Conclusión.}
La combinación de una demostración por inducción estructural y casos de prueba meticulosos proporciona una base sólida para afirmar la correctitud del algoritmo $definirManiobra$. La inducción estructural, que se ha aplicado previamente, asegura que el algoritmo funciona para todos los posibles estados y transiciones del sistema, validando su lógica inherente y estructura recursiva. Por otro lado, los casos de prueba, han sido diseñados para abordar diversas situaciones y escenarios, proporcionando una verificación empírica de la funcionalidad del algoritmo en contextos específicos. Juntos, estos dos enfoques complementan la validación teórica y práctica del algoritmo, asegurando que su diseño a través del $pattern matching$ y ejecución son sólidos y confiables para definir los movimientos que deberían aplicarse para que un tren alcance un estado deseado.
\end{document}