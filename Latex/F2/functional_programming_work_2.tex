\documentclass[12pt, a4paper]{article}
\usepackage{graphicx}
\usepackage{amsmath}
\usepackage{amssymb}
\graphicspath{ {./images/} }
\title{
  \begin{figure}[th]
    \centering
    \includegraphics[width=0.2\textwidth]{Univalle}
  \end{figure}
  \textbf{Universidad del Valle
    \\{\Large Facultad de ingeniería}
  \\{\large Ingeniería en sistemas}}}
\author{Cristian David Pacheco Torres
  \\ 2227437
  \\ Sebastian Molina Cuellar
  \\ 2224491}
\date{28 de Septiembre del 2022}
\begin{document}
\maketitle
Taller 2: Simular un sumador de $n$ digitos a partir de compuertas lógicas sencillas.
\newpage{}
\tableofcontents
\newpage{}
\section{Creación de las compuertas sencillas.}
\subsection{Creando compuertas unarias.}
\subsubsection{Informe de procesos.}
\subsubsection{Informe de corrección.}
\subsubsection{Casos de pruebas.}
\subsection{Creando compuertas binarias.}
\subsubsection{Informe de procesos.}
\subsubsection{Informe de corrección.}
\subsubsection{Casos de pruebas.}
\section{Creando $semisumadores$.}
\subsection{Informe de procesos}
\subsection{Informe de corrección}
\subsection{Casos de pruebas}
\section{Creando $sumadores \ completos$.}
\subsection{Informe de procesos}
\subsection{Informe de corrección}
\subsection{Casos de pruebas}
\section{Construyendo un $sumador-n$}
\subsection{Informe de procesos}
\subsection{Informe de corrección}
\subsection{Casos de pruebas}
\section{Conclusion}
La conclusion
\newpage{}
\begin{displaymath}
 a = \sum F \dot m = \frac{dv}{dt}
\end{displaymath}

\end{document}