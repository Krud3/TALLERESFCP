\documentclass[12pt, a4paper]{article}
\usepackage{graphicx}
\usepackage{amsmath}
\usepackage{amssymb}
\usepackage{listings}
\usepackage{xcolor}
\usepackage{amsfonts}

% Configuración de estilo para el código Scala 
\lstdefinestyle{scalaStyle}{
    language=Scala,                 % Lenguaje del código
    basicstyle=\ttfamily\small,     % Estilo básico de fuente y tamaño
    keywordstyle=\color{blue},      % Estilo para palabras clave
    commentstyle=\color{green},     % Estilo para comentarios
    numbers=left,                   % Mostrar números de línea a la izquierda
    numberstyle=\tiny,              % Estilo de los números de línea
    frame=single,                   % Un marco alrededor del código
    breaklines=true,                % Permitir saltos de línea automáticos
    captionpos=b,                   % Posición de la leyenda
}

\graphicspath{ {./images/} }
\title{
  \begin{figure}[th]
    \centering
    \includegraphics[width=0.2\textwidth]{Univalle}
  \end{figure}
  \textbf{Universidad del Valle
    \\{\Large Facultad de ingeniería}
  \\{\large Ingeniería en sistemas}}}
\author{Cristian David Pacheco Torres
  \\ 2227437
  \\ Juan Sebastian Molina Cuellar
  \\ 2224491}
\date{Septiembre 2023}
\begin{document}
\maketitle
Taller 1
\newpage{}
\begin{abstract}
Your abstract goes here functional programming !!!
\end{abstract}
\newpage{}
\tableofcontents
\newpage{}
\section{Introduction}
Para el desarrollo de este taller, se utilizaron las siguientes funciones en scala: \\
\begin{itemize}
  \item $isEmpty$: Boolean (Devuelve si una lista esta vacia).
  \item $head$: Int (devuelve si una lista $l$ esta vacia).
  \item $tail$: List[Int] (devuelve la lista sin el primer elemento $l$).
  \item $x::l$: devuelve la lista que representa la secuencia $<x,x_1,x_2,...,x_n>$ si $l$ es la lista que representa
  la secuencia $<x_1,x_2,...,x_n>$.
  \item $l1 ++ l2$ devuelve la lista que representa la concatenacion de las secuencias representadas por $l1$ y $l2$.
\end{itemize}
Se da por hecho de que las funciones anteriores estan argumentadas y demostradas, por lo tanto se procede a la resolucion de los ejercicios.
\section{Taller 1 : Recursión}
\subsection{Calcular el tamaño de una lista con un proceso iterativo}
En el listing 1 se puede apreciar la funcion $tamR(l)$ que a traves de un proceso recursivo calcula el tamaño de una lista.
\begin{lstlisting}[style=scalaStyle, caption=Calcula el tamaño de una lista con un proceso recursivo]
  def tamR(l : List[Int] ) : Int = {
    if (l.isEmpty) 0
    else 1 + tamR(l.tail )
  }
\end{lstlisting}
El problema a solucionar es hacer una funcion $tamI(l)$ tal que: \\
$tamR(l) == tamI(l)$
\subsubsection{Informe de procesos}
\textbf{Descripcion de la funcion.}\\
La solucion propuesta se basa en el siguiente algoritmo (ver listing 2.) \\
Para ello se creo la funcion $tamI(l:List[Int]):Int = \{\}$.
\begin{lstlisting}[style=scalaStyle, caption=Calcula el tamaño de una lista con un proceso iterativo]
  def tamI(l : List[Int]): Int = {
    def tam(lst : List[Int], acc : Int): Int = {
      if (lst.isEmpty) acc
      else tam(lst.tail, acc + 1)
    }    
  tam(l, 0)
  } 
\end{lstlisting}
Para esta implementacion se utilizo una funcion auxiliar: \\ $tam(lst:List[Int], acc:Int):Int$ (ver linea 2) la cual recibe como parametros una lista ($l$) y un acumulador ($acc$), el cual se encarga de contar el tamaño de la lista.
En el caso base (ver linea 3), si la lista esta vacia, se retorna el acumulador, en caso contrario se llama a la funcion $tam$ (ver linea 4) con la lista sin el primer elemento ($lst.tail$) y el acumulador incrementado en 1. \\
Para la inicializacion de la funcion $tamI(l:List[Int]):Int$ se llama a la funcion $tam$ (linea 6) con la lista y un acumulador inicializado en 0.\\ \\
\textbf{Tipo de proceso.}\\
Se pretende de que el tipo de proceso es \textbf{iterativo} para ello evaluamos la funcion con parametro $List(12, 3, 1, 8, 4)$ \\
$tamI(List(12, 3, 1, 8, 4))$\\
$\rightarrow  tamI(List(12, 3, 1, 8, 4))$ \\
$\rightarrow tam(List(12,3,1,8,4),0)$\\
$\rightarrow if(List(12,3,1,8,4).isEmpty) ~0 ~else ~tam(List(12,3,1,8,4).tail,0+1)$ \\
$\twoheadrightarrow tam(List(3,1,8,4),1)$\\
$\rightarrow if(List(3,1,8,4).isEmpty) ~1 ~else ~tam(List(3,1,8,4).tail,1+1)$ \\
$\twoheadrightarrow tam(List(1,8,4),2)$\\
$\rightarrow if(List(1,8,4).isEmpty) ~2 ~else ~tam(List(1,8,4).tail,2+1)$ \\
$\twoheadrightarrow tam(List(8,4),3)$\\
$\rightarrow if(List(8,4).isEmpty) ~3 ~else ~tam(List(8,4).tail,3+1)$ \\
$\twoheadrightarrow tam(List(4),4)$\\
$\rightarrow if(List(4).isEmpty) ~4 ~else ~tam(List(4).tail,4+1)$ \\
$\twoheadrightarrow tam(List(),5)$\\
$\rightarrow if(List().isEmpty) ~5 ~else ~tam(List().tail,5+1)$ \\
$\twoheadrightarrow 5$\\
Aunque la funcion utiliza la recursión en su estructura, el proceso que genera es \textbf{iterativo}, ya que el proceso mantiene su forma constante.
\subsubsection{Informe de corrección}
\textbf{Argumentacion sobre la corrección.}\\
Se desea calcular el tamaño de una lista de enteros, tal que si $List(x_1,x_2,x_3,...,x_n)$ $\rightarrow n$, donde $n \in \mathbb{N}_{\geq 0}$ es el tamaño de la lista. Si la lista es vacia entonces $n=0$\\
Sea $L=List(a_1,a_2,a_3,...,a_n)$, $n\geq0$. \\
Sea $f:L \rightarrow \mathbb{N}_{\geq 0}$, $f(L)=n$\\
Y sea $P_f$ (ver Listing 2) la propiedad que se desea demostrar, $P_f(L):f(L)=n$.\\
\begin{itemize}
  \item Un estado $s=(l,acc)$ donde $l=List(a_i,a_{i+1},...,a_n)$ es una cola de $L$.
  \item El estado inicial $s_0=(L,0)=(List(x_1,x_2,x_3,...,x_n),0)$.
  \item $s = (l,acc)$ es final si $l$ es vacia.
  \item $Inv(l,acc)\equiv l = List(a_i,a_{i+1},...,a_n) \wedge acc = f(List(a_1,a_2,...,a_{i-1}))$.
  \item $transformar(l,acc) = (l',acc')$ donde $l'=l.tail$ y $acc'=acc+1$. 
\end{itemize}
\textbf{Demostracion.}\\
\begin{enumerate}
  \item $Inv(s_0)$: el estado inicial cumple la condición invariante. Como $s_0=(L,0)$, entonces $l=L$ y $acc=0$, por lo tanto $l=List(a_1,a_2,...,a_n)$ y $acc=f(List())=0$, por lo tanto $Inv(s_0)$ se cumple.% revisar esta
  \item ()
\end{enumerate}
%Para demostrar la correccion de la funcion $tamI(l:List[Int]):Int$ se utilizo el principio de induccion sobre el tamaño de la lista $l$.\\
%Caso base.\\
%Si $l$ es una lista vacia, entonces $tamI(l) == tamR(l) == 0$\\
%\textbf{Hipotesis de induccion.}\\
%Supongamos que $tamI(l) == tamR(l)$ para toda lista $l$ de tamaño $n$.\\
%\textbf{Paso inductivo.}\\
%Sea $l$ una lista de tamaño $n+1$, entonces $l$ puede ser representada como $x::l'$, donde $x$ es el primer elemento de la lista y $l'$ es la lista sin el primer elemento.\\
%Por lo tanto $tamI(l) == tamI(x::l') == tamI(l') + 1$\\
%Por hipotesis de induccion $tamI(l') == tamR(l')$\\
%Por lo tanto $tamI(l) == tamR(l) == tamR(l') + 1 == tamR(x::l')$\\
%\textbf{Conclusion.}\\
%Por principio de induccion, se concluye que $tamI(l) == tamR(l)$ para toda lista $l$.

\subsection{Dividiendo una lista en dos sublistas a partir de un pivote}
\subsubsection{Informe de procesos}
\subsubsection{Informe de corrección}
\subsection{Calculando el k-ésimo elemento de una lista}
\subsubsection{Informe de procesos}
\subsubsection{Informe de corrección}
\subsection{Ordenando una lista}
\subsubsection{Informe de procesos}
\subsubsection{Informe de corrección}
\section{Conclusion}
La conclusion
\newpage{}
\begin{displaymath}
 a = \sum F \dot m = \frac{dv}{dt}
\end{displaymath}

\end{document}
