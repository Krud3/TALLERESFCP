\documentclass[12pt, a4paper]{article}
\usepackage{graphicx}
\usepackage{amsmath}
\usepackage{amssymb}
\usepackage{float}
\usepackage{listings}
\usepackage{xcolor} 
\graphicspath{ {./images/} }
\title{
  \begin{figure}[th]
    \centering
    \includegraphics[width=0.2\textwidth]{Univalle}
  \end{figure}
  \textbf{Universidad del Valle
    \\{\Large Facultad de ingeniería}
  \\{\large Ingeniería en sistemas}}}
\author{Cristian David Pacheco Torres
  \\ 2227437
  \\ Juan Sebastian Molina Cuellar
  \\ 2224491}
\date{\today}

\begin{document}
\maketitle
{Taller 4: Colecciones y Expresiones For:
\\ El problema de la subsecuencia incremental de longitud máxima}
\newpage{}

\tableofcontents
\newpage{}
\section{Solución ingenua usando fuerza bruta}
\subsection{Generación de los índices asociados a todas las subsecuencias}
\subsubsection{Informe de uso de colecciones y expresiones for}
\begin{table}[H]
    \scriptsize
   \begin{tabular}{ |p{4cm}|p{3cm}|p{5.5cm}|  }
    \hline
    \multicolumn{3}{|c|}{Tabla base completar} \\
    \hline
    \textbf{Función}& ¿Se utilizó colecciones y expresiones for?  & ¿Razón?\\
    \hline
     Completar... & Completar... &  Completar... \\
     \hline
   \end{tabular}
   \centering
   \caption{Completar...}
   \end{table}
\subsubsection{Informe de corrección}
\textbf{Argumentación sobre la corrección: \\}
\textbf{Casos de prueba: \\}
\subsubsection{Conclusiones}
\subsection{Generación de todas las subsecuencias de una secuencia}
\subsubsection{Informe de uso de colecciones y expresiones for}
\begin{table}[H]
    \scriptsize
   \begin{tabular}{ |p{4cm}|p{3cm}|p{5.5cm}|  }
    \hline
    \multicolumn{3}{|c|}{Tabla base completar} \\
    \hline
    \textbf{Función}& ¿Se utilizó colecciones y expresiones for?  & ¿Razón?\\
    \hline
     Completar... & Completar... &  Completar... \\
     \hline
   \end{tabular}
   \centering
   \caption{Completar...}
   \end{table}
\subsubsection{Informe de corrección}
\textbf{Argumentación sobre la corrección: \\}
\textbf{Casos de prueba: \\}
\subsubsection{Conclusiones}
\subsection{Generación de todas las subsecuencias incrementales de una secuencia}
\subsubsection{Informe de uso de colecciones y expresiones for}
\begin{table}[H]
    \scriptsize
   \begin{tabular}{ |p{4cm}|p{3cm}|p{5.5cm}|  }
    \hline
    \multicolumn{3}{|c|}{Tabla base completar} \\
    \hline
    \textbf{Función}& ¿Se utilizó colecciones y expresiones for?  & ¿Razón?\\
    \hline
     Completar... & Completar... &  Completar... \\
     \hline
   \end{tabular}
   \centering
   \caption{Completar...}
   \end{table}
\subsubsection{Informe de corrección}
\textbf{Argumentación sobre la corrección: \\}
\textbf{Casos de prueba: \\}
\subsubsection{Conclusiones}
\subsection{Hallar la subsecuencia incremental más larga}
\subsubsection{Informe de uso de colecciones y expresiones for}
\begin{table}[H]
    \scriptsize
   \begin{tabular}{ |p{4cm}|p{3cm}|p{5.5cm}|  }
    \hline
    \multicolumn{3}{|c|}{Tabla base completar} \\
    \hline
    \textbf{Función}& ¿Se utilizó colecciones y expresiones for?  & ¿Razón?\\
    \hline
     Completar... & Completar... &  Completar... \\
     \hline
   \end{tabular}
   \centering
   \caption{Completar...}
   \end{table}
\subsubsection{Informe de corrección}
\textbf{Argumentación sobre la corrección: \\}
\textbf{Casos de prueba: \\}
\subsubsection{Conclusiones}
\section{Hacia una solución más eficiente}
\subsection{Calculando $SIML_i(S)$}
\subsubsection{Informe de uso de colecciones y expresiones for}
\begin{table}[H]
    \scriptsize
   \begin{tabular}{ |p{4cm}|p{3cm}|p{5.5cm}|  }
    \hline
    \multicolumn{3}{|c|}{Tabla base completar} \\
    \hline
    \textbf{Función}& ¿Se utilizó colecciones y expresiones for?  & ¿Razón?\\
    \hline
     Completar... & Completar... &  Completar... \\
     \hline
   \end{tabular}
   \centering
   \caption{Completar...}
   \end{table}
\subsubsection{Informe de corrección}
\textbf{Argumentación sobre la corrección: \\}
\textbf{Casos de prueba: \\}
\subsubsection{Conclusiones}
\subsection{Calculando una subsecuencia incremental más larga, versión 2}
\subsubsection{Informe de uso de colecciones y expresiones for}
\begin{table}[H]
    \scriptsize
   \begin{tabular}{ |p{4cm}|p{3cm}|p{5.5cm}|  }
    \hline
    \multicolumn{3}{|c|}{Tabla base completar} \\
    \hline
    \textbf{Función}& ¿Se utilizó colecciones y expresiones for?  & ¿Razón?\\
    \hline
     Completar... & Completar... &  Completar... \\
     \hline
   \end{tabular}
   \centering
   \caption{Completar...}
   \end{table}
\subsubsection{Informe de corrección}
\textbf{Argumentación sobre la corrección: \\}
\textbf{Casos de prueba: \\}
\subsubsection{Conclusiones}

\end{document}