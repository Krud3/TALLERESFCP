\documentclass[12pt, a4paper]{article}
\usepackage{graphicx}
\usepackage{amsmath}
\usepackage{amssymb}
\graphicspath{ {./images/} }
\title{
  \begin{figure}[th]
    \centering
    \includegraphics[width=0.2\textwidth]{Univalle}
  \end{figure}
  \textbf{Universidad del Valle
    \\{\Large Facultad de ingeniería}
  \\{\large Ingeniería en sistemas}}}
\author{Cristian David Pacheco Torres
  \\ 2227437
  \\ Sebastian Molina Cuellar
  \\ xxxxx}
\date{Agosto 2022}
\begin{document}
\maketitle
Taller 1
\newpage{}
\begin{abstract}
Your abstract goes here functional programming
\end{abstract}
\newpage{}
\tableofcontents
\newpage{}
\section{Introduction}
A introduction
\section{Solucion ejercicios de programación funcional}
\subsection{Calcular el tamaño de una lista con un proceso iterativo}
\subsubsection{Proceso}
\subsubsection{Corrección}
\subsubsection{Casos de pruebas}
\subsection{Dividiendo una lista en dos sublistas a partir de un pivote}
\subsubsection{Proceso}
\subsubsection{Corrección}
\subsubsection{Casos de pruebas}
\subsection{Calculando el k-ésimo elemento de una lista}
\subsubsection{Proceso}
\subsubsection{Corrección}
\subsubsection{Casos de pruebas}
\section{Conclusion}
La conclusion
\newpage{}
\begin{displaymath}
 a = \sum F \dot m = \frac{dv}{dt}
\end{displaymath}

\end{document}
